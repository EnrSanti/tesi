\documentclass[a4paper,11pt]{report} %definisce classe documento e dim font
	
	\usepackage{subcaption}
	\usepackage{graphicx} 
	\usepackage{hyperref}
	\usepackage{xcolor}
	\usepackage{titlesec}
	\usepackage{amsmath}
	\usepackage{anyfontsize}
	\usepackage{amssymb}
	\usepackage{amsthm}
		\titleformat{\chapter}[block]
	  {\normalfont\LARGE\bfseries}{\hspace*{\fill}{\fontsize{60pt}{96pt}\selectfont \underline {$\overline	{{\thechapter}}$}}}{0.0em\newline\newline}{\LARGE}
	\titlespacing*{\chapter}{0pt}{-35pt}{10pt}
	
	\graphicspath{ {./imgTesi/} }
	\title{\textbf{Draft tesi (\textcolor{orange}{latest major changes in orange})}}
	\author{Enrico Santi}
	
	\newtheorem{exmp}{Example}
\begin{document}

%per non separare 2 parole da \n -> parola1~parola2
%specifica dove split parola -> par\-ola

\maketitle

\thispagestyle{empty}
\vspace*{-15.5em}
\hspace*{-14.5em}
\includegraphics[scale=1.05]{frontespizioOK.pdf}

\clearpage
\newpage 
\pagenumbering{gobble}
\ % empty page

\newpage

\pagenumbering{arabic}
\setcounter{page}{1}

\tableofcontents



\newpage
%fine pag
\chapter{Introduction}
This thesis mainly concerns the work carried out during the internship at the CD-Lab at the Department of Mathematics, Computer Science and Physics of the University of Udine.
%inserisci cosa
The internship has been focused on the development of an extension of the graphical user interface (GUI) of MatCont, a MATLAB software for continuation and bifurcation of ordinary differential equations (ODEs). 
The aim is to provide a user-friendly GUI to study dynamical systems described by delay differential equations (DDEs),  which arise in many application fields, i.e. control theory, ecology and epidemiology.
The thesis is organized as follows.
\newline In Chapter \ref{cap2} after a brief introduction on differential equations and dynamical systems we introduce the discretization approach, which is the theoretical basis of the MatCont extension, to approximate DDEs by a system of ODEs.\newline
Chapter \ref{cap6} we describe the fundamental aspects of the current version of MatCont's GUI and how it works while in in Chapter \ref{cap7} we will focus on the new capabilities of the extended version of MatCont, what are the new capabilities of this software and how the development has been carried out. Finally in Chapter \ref{conclusions} we conclude with some remarks on the future additional tools to implement to analyze the dynamics of renewal equations too.
\newline
\newline
The work done and described in Chapter \ref{cap7} will also be presented at the \emph{17th IFAC Workshop on Time Delay Systems (TDS)}, September 2022 in Montreal, Canada. 

\clearpage
\chapter{Basics on dynamical systems}
\label{cap2}

%inserire intro

\section{Dynamical systems}

%qui singolare
A dynamical system describes the evolution in time of phenomenon and finds application in several fields, from engineering to biology (\cite{arrowsmith1990introduction,kuznetsov}).
Mathematically a phenomenon is represented by a state variable and the dynamical system is defined as a triple $\{T,X,\phi^t\}$ where $T$ is the time set, $X$ is the state space\footnote{all the values describing the state the considered dynamical system can assume as the time progresses.}, and \{$\phi^t$\}$_{t \in T}$ is the family of evolution operators $\phi^t : X \rightarrow X$ satisfying: 
\begin{itemize}
\item$\phi^0=id$, where $id$ is the identity on $X$, 
\item$\phi^{t+s}=\phi^t \circ \phi^s$.
\end{itemize}
%continua: The first property of the evolution operators
A first major distinction is related to the time set $T$:
\begin{itemize}
\item when $ T \subset {\rm I\!N} $ we call them discrete dynamical systems,
\item when $ T $ is a more general subset of ${\rm I\!R}$, such as an interval, we call them continuous dynamical systems.

\end{itemize}
As described in \cite{MEERSCHAERT2013115} the relationship between discrete and continuous systems recalls the relationship between $\Delta x/\Delta t$ and $dx/dt$.
In this thesis we will treat only continuous dynamical systems which are described by ordinary differential equations (ODEs) and delay differential equations (DDEs), from here after we refer to them simply as dynamical systems.

%In some sense when modelling a phenomenon with a discrete dynamical system system we are only interested in the states of the system at discrete points in time \cite{barbarossa2011stability}. Another difference from a continuous-time dynamical system is that the description of such a system is not based on ODEs or delay equations but on recursive equations. A simple example of continuous dynamical system is represented by the Malthus equation:  
%\begin{equation}
%y'(t)=r y(t) 
%\label{malthusEquation}
%\end{equation}
%The time-discrete counterpart system of the system described by \eqref{malthusEquation} is:

%\begin{equation} 
%\label{malthusdiscrete}
%y_{t+1}=y_t+ry_t
%\end{equation}

%\vspace{0.8em}




\section{Ordinary differential equations}



%An ordinary differential equation is a relation involving one or more derivatives of an \emph{unknown} function and the function itself \cite{odeIntro}. 
We consider autonomous differential equations of the form:
\begin{equation}
\label{odeeq}
y'(t)=F(y(t)), \;\;\; t \geq 0
\end{equation}
where $t$ is the independent variable, $y$ the unknown function ${\rm I\!R}^d$-valued with $d\geq1$   and $F : A \subseteq  {\rm I\!R}^{d} \rightarrow  {\rm I\!R}^d$.
Solving a differential equation (when it is possible) means finding a set of functions satisfying the relationship on a suitable interval. To obtain a unique solution of \eqref{odeeq} additional conditions are needed. So we consider the following initial value problem (IVP):
\begin{equation} 
\begin{cases}
\label{IVPode}
y'(t)=F(y(t)),\;\;\; t \geq 0\\
y(t_0)=y_0 \\
\end{cases}\,.
\end{equation}
which under suitable regularity assumptions on the right hand side $F$ has a unique solution $y(t)=\phi^t(y_0)$, for $t \in [0,t_f]$. Moreover it can be proven for \eqref{IVPode} that the solution continuously depends on the initial value $y_0$ \cite{chicone}.The equation \ref{odeeq} defines a dynamical system where $y(t) \in  {\rm I\!R}^d$ is the state at time $t$, $X \in {\rm I\!R}^d$ is the state space and $T=[0,tf]$. Note that the state space is ${\rm I\!R}^d$ so the equation \eqref{odeeq} defines a finite-dimensional dynamical system.
\newline
A trajectory of a solution of an IVP consists of following set:
\begin{equation} 
\{(t,y(t)): t \in T\}
\end{equation}
%modifica
Related to this concept, one of the first geometric visualizations for ODEs is the trajectory diagram. In relation to a given IVP, it presents on one of its axes the independent variable, and on the other the component considered, for what concerns systems of ODEs (with $d > 1$) usually each component ($y_i$, unknown function) is considered separately in a different diagram. This diagram in relation to an ODE presents the related vector field (i.e. an assignment of a vector to each point in a given space), where at each point, the slope of the corresponding vector represents the value of the first derivative of the component considered at that specific point. This geometric visualization allows to qualitatively study different IVPs without analytically solving them, giving insights on the problem \cite{chicone}. In Figure \ref{fig:malthus} and \ref{fig:logistic} we show an example of these diagrams in the case of two scalar ($d=1$) equations. In the following two examples the solutions can be calculated exactly but this is not always possible, for this reason usually numerical softwares and integrators are used (i.e. MATLAB).
\newline
\begin{exmp} 

The (continous)\footnote{its discrete counterpart is described by the recursive equation $y(t+1) = y(t) + ry(t)$} Malthus equation is defined as follows:
\begin{equation} 
\label{malthuseq}
y'(t)=ry(t),\;\;\; t \geq 0\\
\end{equation} 
It is a linear autonomous first order ODE, in which $r$ represents a parameter, that can be solved analytically by separation of variables\footnote{A procedure for solving ODEs that can be rewritten as $y'(t)=a(t)b(y)$}.\newline
The general solution of the equation is $\phi^t(y_0)=y_0 e^{rt}$ for $t \geq 0$ where $y_0$ is the initial value chosen for $y(0)$.\newline
An equilibrium, a constant solution on the domain, for the equation (regardless the value of the parameter $r$) is obtained with the initial value $y_0=0$ at $t_0=0$. Plotting the solution of the relevant IVP (with different values of $r$) will always result in a straight line on the trajectory plane.

\end{exmp}
\begin{exmp}

The logistic model is described by the logistic equation:
\begin{equation} 
\label{logisticEQ}
y'(t)=ry(t)(1-\frac{y(t)}{K}),\;\;\; t \geq 0\\
\end{equation}
This model describes the evolution of a population of individuals where the parameter $r$ is the rate of maximum population growth (i.e. the rate at which the individuals can reproduce)\cite{weisstein2003logistic} and it is usually considered a positive value and $K$ represents the maximum number of individuals the environment can sustain (the capacity).  
The general solution of the equation \ref{logisticEQ} (where $y_0$ is the value of $y(0)$ in the related IVP) can be obtained by separation of variables:\newline
\begin{equation}  
\label{logisticEQsol}
\phi^t(y_0)=\frac{K}{1+(\frac{K}{y_0}-1)e^{-r(t)}},\;\;\; t \geq 0\\
\end{equation}

The system presents two equilibria: $e_0=0$ and $e_1=K$. For what concerns positive values of $r$, looking at Figure \ref{fig:logistic} it can be seen that all the solutions approach to $e_1$ (it is asymptotically stable) while they move away from the trivial equilibrium ($e_0$ is asymptotically unstable). 
\end{exmp}

%Another distinction can be made on whether or not the relation describing the equation depends not only on the function but also on the independent variables of the function. In the case the relation doesn't depend on the independent variables the differential equation is said to be \emph{autonomous}.\newline
%The first example below shows an autonomous ordinary differential equation while the second example represents a nonautonomous one:


%\begin{center}
%$y'(t)+y(t)+1=0$,
%\end{center}
%\begin{center}
%$y'(t)+y(t)+t=0$,
%\end{center}
%A more concrete difference concerning these two kinds of ODEs is presented in Paragraph \ref{geometric}. 
%\newline
%Another important distinction is on linear or nonlinear differential equations. Linear differential equations can be described by a linear function where the variables are the unknown functions \cite{chicone}.%(i.e. linear differential equations are defined by homogeneous polynomials of degree one in terms of the unknown functions).
%\newline\newline
%Given an ordinary differential equation, or ODE, the domain of the unknown function is then by definition an interval $I \subset {\rm I\!R}$. While it's true that other functions defined over a generic $A \subset {\rm I\!R}$ that satisfy the relationship on the required interval exist, they aren't taken into account by the classical definition of solution. The maximum order of derivative of an ODE is referred to as the order of the equation.
%A useful observation that could be used when studying ODEs is that given an ODE of degree $n$ it can be re-written as a system of $n$ first order ODEs. \textcolor{orange}{ }.\newline \newline
%\subsection{Initial value problem}\label{IVP}
%As written in the introduction, solving a differential %equation or a system of equations (i.e. looking for the %\emph{general} solution) means finding a class/family of %functions.
%Usually in addition to finding the general solution of an ODE it may be required to look for a specific solution (i.e. a single function that satisfies the relationship with some additional conditions in the domain of the unknown function).
%An initial condition is an equation involving the unknown function (or one of its derivatives) evaluated at a specific point $t_0 \in I$ (usually and also for this thesis $t_0=0$) and a value in the codomain of the unknown function (or it's derivatives).
%The initial value problem for a given equation can be viewed as a system containing the differential equations considered and $n$ additional equations that specify the initial conditions (at $t_0$) where $n$ is the order of the equation.
%Let $y^{(n)}$ be an ODE of order $n$, the initial conditions associated to the IVP will resemble the following system:
%\begin{center} $
%\begin{cases}
%y(t_0)=y_0 \\
%y'(t_0)=y_1 \\
%...\\
%y^{(n-1)}(t_0)=y_{n-1} 
%\end{cases}\,.$
%\end{center}

%{inserire: non verrà ripotata la dimostrazione di esistemza e unicità o un maggiore formalismo}
%As in many other mathematical fields a geometrical interpretation of the studied concepts helps to obtain a better understanding of the topic. 


\begin{figure}[htp]
\centering
\includegraphics[scale=0.35]{imgTesi/malthusTrajectory.png}
\caption{The trajectory diagram of the malthus equation, with $r=4$, showing two solutions.}
\label{fig:malthus}
\end{figure}

\begin{figure}[htp]
\centering
\includegraphics[scale=0.35]{imgTesi/logisticTrajectory.png}
\caption{The trajectory diagram of the logistic equation, with $K=3$ and $r=2$ showing two solutions.}
\label{fig:logistic}
\end{figure}


The two trajectory diagrams in Figure \ref{fig:malthus} and \ref{fig:logistic} are numerically obtained. They show two solutions of the respective \eqref{malthuseq} and \eqref{logisticEQ} initial value problems each, with $y(t_0=0)=1$ and $y(t_0=0)=0$ for the Malthus equation and $y(t_0=0)=1$ and $y(t_0=0)=1$ for the Logistic equation.
\newline
Two other fundamental concepts are the orbit of a solution of an ODE and the phase portrait. An orbit in an interval (of the independent variable) can be defined as a subset of the state space such that each value is assumed by the solution in the interval (i.e. $\{(y(t)): t \in T\}$) \cite{chicone}. On the other hand the phase portrait is a partitioning of the state space into orbits \cite{kuznetsov}.
For what concerns ODEs defined as \ref{odeeq} also the \emph{phase} or \emph{orbit} plane is presented, this plane present on its axes some (two or three, according to the number of axes) of the solutions of the system. Since $d$ may be greater than three, more phase planes can be presented for a system (depending on the solutions taken into account). This plane doesn't present the independent variable axis (e.g. the time axis), allowing to study the dynamics of a system independently from the time perspective (i.e. study the orbits).
\newline
From the phase plane of a given IVP (with proper initial conditions) we can observe curves (i.e. single orbits) that show some interesting properties, for example closed curves and single points denote special orbits, respectively referred to as a \emph{periodic orbit} and a \emph{steady state} or an \emph{equilibrium}. In Chapter \ref{stability} the related concepts of stability and bifurcation are briefly discussed. The phase diagram can also be plotted for an  ODE with $d=1$, but this will result in a one dimensional line.\newline

%scrivi equilibri

%%fine



\section{Delay differential equations}

With reference to \cite{diekmann2012delay} {\it
“a delay equation is a rule for extending a function of time towards the future on the basis of the (assumed to be) known history”}.
Let $y$ denote a function of time, then the history function is defined as:

\begin{equation}
y_t(\theta)=y(t+\theta),\;\;\;  \theta \in [-\tau,0]
\end{equation}
where $\tau >0$ is the bounded delay and $[-\tau,0]$ is the delay interval. 
A DDE is defined as:
\begin{equation}\label{DDE}
y'(t)=F(y_t), \;\;\; t \geq 0
\end{equation}
where $F: C([-\tau, 0]; {\rm I\!R}^d) \to {\rm I\!R}^d$ is a nonlinear map \cite{diekmann2012delay}.
\newline 
\newline 
A first useful distinction, that will be helpful in Chapter \ref{cap7}, is distinguishing between discrete and distributed delays. In the first case we have that $F(y_t)= f(y_t(\tau_1), y_t(\tau_2),…,)$ with $0 \leq \tau_1 < \tau_2<…\leq \tau$, while an example of DDEs with distributed delay is $F(y_t)=\int_{-\tau}^{0}{f(y(t+\theta))}{d\theta}$.
\newline
In order to define an IVP for \eqref{DDE} it’s not sufficient to specify an initial vector, but it is necessary to give an initial function $\psi$. So an IVP reads as: 
\begin{equation}
\label{ddesystem}
\begin{cases}
y'(t)=F(y_t),   \;\;\; t\geq0\\
y(\theta)=\psi(\theta), \;\; \theta \in [-\tau,0]\\
\end{cases}\,.
\end{equation}
Under suitable assumption of $F,$ it can be proved that \eqref{ddesystem} has a unique continuous solution in $[-\tau,t_f]$ \cite{diekmann2012delay}. 
A DDE defines a infinite-dimensional dynamical system where the state at time $t\geq 0$ is the history function $y_t$ and the state space is $X= C([-\tau, 0]; {\rm I\!R}^d)$.
For example the following equation, where $\tau_1$ is a parameter, presents two discrete delays and $\tau=\max\{\tau_1,1\}$:
\begin{equation} 
z'(t)=z(t-1)+z(t-\tau_1) 
\end{equation}
\section{Stability and bifurcations}
\label{stability}
In terms of differential equations, stability concerns the analysis of the effect of small perturbations on a given solution \cite{chicone}.
Intuitively a solution is said to be stable if any other solutions of the differential equation close to it remains close to it as the system evolves. If a solution doesn’t have this property it is called unstable.
Moreover a stable solution is asymptotically stable if distance between the given solution and the solutions close to it approaches zero as time goes to infinity. From the point of view of dynamical systems, the study of stability properties of equilibria and periodic solutions are important tasks.
\newline\newline
 When taking into account differential equations
 depending on one or more parameters such as \eqref{malthuseq}, it is interesting to study when a small change of the value of one parameter causes a “qualitative” change in the behavior of the system. Such values of the parameters are known as “bifurcation”
 points. Depending on which aspect of the system's solution changes, they can be classified into different classes as described in \cite{kuznetsov}. For instance a Hopf bifurcation is a critical value of the parameter where an equilibrium loses its stability
 and a stable periodic orbit arises.  
 \newline
We consider differential equations which depend on parameters defined as: 
\begin{equation}
y'(t)=F(p,y(t)), \;\;\; t \geq 0
\end{equation}
\vspace{0.8em}
where $p \in {\rm I\!R}^m$ represents the vector of
the $m$ parameters.
 \newline
As reported in the introduction, MatCont is a collection of Matlab routines for
 the stability and bifurcation analysis of parametrized ODEs. It is used to study dynamical systems defined by ODEs, while \ref{why} other softwares such as DDEBiftool, can deal with dynamical systems defined by DDEs. These softwares though, are usually less complete than their ODE counterparts. For example the method presented (from \cite{DDEdisc}) in section \ref{back} allows also to deal with distributed delays making possible for the user to study different kinds of systems, this is not the case with some softwares for DDEs.
%ddebiftool usa solo ritardi discreti. con questo metodo.
\subsubsection{Why discretizing to an ODE system}
\label{why}
%modifica sotto stability
In the next chapter a technique to study DDE systems by approximating them with an ODE system is recalled from \cite{DDEdisc}. There are two main reasons behind the discretization of a dynamical system described by delay equations to an ODE system:
\begin{itemize}
\item We are only able to study with the help of a computer finite dimensional objects and it wouldn't be possible to deal with the original problem since a DDE describes infinite-dimensional dynamical system. 
\item While a few softwares to carry out bifurcation analysis of delayed equations exist (for example DDEBiftool which can treat DDEs with discrete and state dependent delays), softwares for the study of ODE systems are generally more complete and the methods utilized more known.
\end{itemize}



\section{From DDEs back to ODEs }
\label{back}
As the first step to derive a system of ODEs from a system of DDEs, we rewrite the DDE as an abstract differential equation on the state space $X$, i.e.:
\begin{equation} 
\label{adesys}
v'(t)=A_F(v(t)),  \;\;\; t\geq0\\
\end{equation}
where $A_F$: $dom(A_F) \subset X$ $\to X$ the infinite-dimensional operator called generator \cite{DDEdisc} given by: 
\begin{equation} 
A_F(\psi)=\psi',\;\;\;
dom(A_F)=\{ \psi \in X : \;\psi' \in X,\; \psi'(0)=F(\psi)\}
\end{equation} 
The idea is to approximate the state $v(t)$ with its interpolating polynomial at selected nodes in the delay interval, and to apply the operator $A_F$ to it. Note that the interpolating polynomial is described by the value of the $v(t)$  at the interpolating nodes. 
Given the discretization parameter $M \in {\rm I\!N}\backslash\{0\}$, we select now the $M+1$ Chebyshev extrema on the interval $[-\theta,0]$: $\theta_M= - \tau <...<\theta_0=0$. 
\newline
By defining the restriction operator:
\begin{equation} 
R_M(\psi)=(\psi(\theta_1);...;\psi(\theta_M);) \in {\rm I\!R}^{d(M)}, \; \psi \in X
\end{equation}
and the prolongation operator:
\begin {equation}
P_M(\Psi_M)(\theta)=\sum_{i=0}^{M} \ell_{M,i}(\theta)\;\Psi_{M,i},  \;\;\;\;\; \theta  \in [-\tau, 0], \; \Psi_{M} \in R^{d(M+1)}
\end{equation}
we can construct the finite dimensional operator on ${\rm I\!R}^{d(M+1)}$ given by:  
\begin{equation} 
A_{F,M}(\Psi_M)=(F(P_M(\Psi_M));R_M'(P_M(\Psi_M)))
\end{equation}
Finally the abstract differential equation \eqref{adesys} is turned into the following ODE:
\vspace{0.5em}
\begin{equation}
\frac{d}{dt}(v_M(t))=A_{F,M}(v_M(t)),  \;\;\; t\geq0\\
\end{equation}
\vspace{0.5em}
which dynamics mimics the dynamics of the original system.

 

\chapter{A brief introduction to MatCont}
\label{cap6}
MatCont is a software, based upon MATLAB, that provides routines and a GUI to study the behavior of continuous dynamical systems defined by ODEs, particularly for what concerns the bifurcation analysis\cite{documentazione}.
\newline
The GUI and routines provided are independent, since the functionalities of MatCont can also be used from MATLAB's command line.
\newline
In the following paragraphs the main aspects of MatCont's GUI and how the user could interact with it will be discussed, they should be kept in consideration when the extension part will be described in the next chapter.
There will not be an exhaustive description of the parsing done by MatCont on the data inserted by the user since during the extension process new parsing routines were introducted without interfering with the pre-existing ones.
%sposta su e ribadisci punto 2

\section{Overview of the current MatCont GUI}

\subsection{MatCont Main Window}
MatCont presents to the user a minimalist but rather intuitive interface. 
The main window shown to the user is divided into three sections: \emph{Class}, \emph{Current System} and \emph{Current Curve}.
\emph{Class} shows the user what type of system is currently loaded (only ODE in the official version), while \emph{Current System} shows the user some basic information about the system itself, such as the name and how the derivatives will be computed (i.e. N for numerically, S for symbolically, and R for from window). \emph{Current Curve} shows the name of curve loaded and what is going to be computed, to give the user some recall of what the state of the software is. In this section, information such as the properties of the initial point chosen (i.e. ``initial point type'' ) and the method that's going to be executed when computing the curve (i.e. ``initializer'') are displayed.
Above these three sections a toolbar menu is presented, from here the user can create a new system, load, edit or delete an already existing one. The type of the specified initial point can also be selected from this menu (e.g. a point without special properties, an equilibrium etc...) and the same is true for the curve to compute (e.g. an orbit, an equilibrium etc...). The user from this toolbar menu can also open an output window, both numerical and graphical (i.e. a Euclidean plane with two or three axes). The graphical output window will plot the graph (along the specified axes) of what is being computed, at the same time the numerical output window will show the values of the specified parameters, coordinates and other quantities of interest at every step of the computation. 

\begin{figure}[htp]
\centering
\includegraphics[scale=0.500]{imgTesi/mainMatCont.png}
\caption{MatCont's main window}
\label{fig:mainMatCont2}
\end{figure}

\subsection{Inserting a new system}
By pressing on Select, System, New in the toolbar menu the user can have access to a new window called \emph{System} from which MatCont allows the user to insert a new system. \newline This window presents the user several input fields to be filled in:
\begin{itemize}
	\item Name: Allows the user to specify the name of the system. 
	\item Coordinates: Must contain every coordinate of the system.
	\item Parameters: Must contain all the parameters used in the system. 
	\item Time: Must contain the name of the independent variable on which the equations in the system depend.
\end{itemize}

The name of the system must not contain spaces, while the coordinates and the parameters have to be separated by commas.
\newline
There is then a table with radio buttons that specify for each order of derivative how it should be computed, \emph{Numerically} the value of the derivative will be numerically approximated, \emph{From window} will allow the user to insert the derivative for each equation in the system (this can be done to save time during the computation) and \emph{Symbolically} which will specify to compute the derivative symbolically (only if the Symbolic toolbox is installed in MATLAB).
At the bottom of this window a larger text entry window is shown to allow the user to type the system in\label{systemWindow}. The syntax a system must satisfy requires writing every equation such that the left hand side presents just the first derivative of the coordinate considered. The time dependence is omitted and every operator must be explicitly written.

\paragraph{What happens next?} Let for example the name of the new system inserted be ``NewTypedSystem'', when the user has correctly inserted all the required data and pressed the \emph{Ok} button MatCont will parse the data entered and create two files under the folder \emph{System} named ``NewTypedSystem.m'' and ``NewTypedSystem.mat'' \cite{documentazione}.
These files are respectively a MATLAB script file, and a binary file. The latter will contain a struct with 36 fields describing the system, such as the list of coordinates, the list of parameters, the equations in the system, the dimension of the system (i.e. the number of different coordinates) etc...\newline
This binary file contains the struct that will be loaded in order to initialize the fields of the current struct representing the system when ``Edit'' or ``Load'' is pressed. The .mat file, on the other hand, contains routines such as \emph{fun\_eval} and \emph{init} that will be called by MatCont when performing different computations on the system.


\begin{figure}[htp]
\hspace*{-3.2em}
\includegraphics[scale=1.5]{imgTesi/matcont_system_state_diagram2.jpg}
\caption{The state diagram regarding the insertion of a new system}
\label{stateDiagram}
\end{figure}

\newpage
\subsection{Editing an existing system}
By pressing on \emph{Select}, \emph{System}, \emph{Load/Edit/Delete System} the user will be presented a new window in which the list of existing systems and a list of buttons including \emph{Load}, \emph{Delete} or \emph{Edit} are shown. \newline 
When the desired system is selected and ``Edit'' is pressed the same window used to input the new system is shown, this time though, the .m file related to the system will be loaded and all the fields will be already filled with the system details. The user can then modify what needed and press the \emph{Ok} button to confirm the changes made, now both the .m and .mat file will be overwritten with the new information and the system will be reloaded from them. Observe that all the previously computed and saved curves will be deleted by this process.

\section{The existing documentation}
On the page from which MatCont can be downloaded it is possible to find some guides and tutorials that introduce MatCont to a new user\footnote{https://sourceforge.net/projects/matcont/files/Documentation/MatContODE/Tutorials7px/} by presenting a series of examples that can be repeated step by step. A manual also exists that provides more knowledge on how this software works and how it is structured\footnote{https://sourceforge.net/projects/matcont/files/Documentation/MatContODE/ManualAug2019.pdf}.	
\newline  
\label{documentation}On the other hand the code would considerably benefit from the insertion of comments or method contracts, especially in some regions of the code, which may be quite difficult to read and subsequently modify.

\chapter{Extending the user interface of MatCont}
\label{cap7}
In this chapter will be presented how the interface of MatCont has been extended, how this task has been achieved, some difficulties faced and what are now the software capabilities. 


\section{MatCont's extended capabilities}
The capabilities of the extended MatCont version (i.e. what systems of differential equations can recognise) have been gradually increased since an iterative development process was used, see section \ref{iterative}.
The systems that can be studied with MatCont now include all the previous ODE systems (referred in later paragraphs as \emph{standard ODE systems}) as well as DDE systems with one or more discrete delays, such as the Mackey-Glass equation,$y'(t)=\beta\cdot y(t-\tau)/(1+y(t-\tau)^n)-\gamma\cdot y(t)$, and systems with (finite) distributed delays as for example $y'(t)=\int_{-1}^{0} k\cdot y(t+z) dz$.
Both the integration extremes in distributed delays must fall within one of the following classes:

\begin{itemize}
    \item A constant (e.g. $\int_{-3}^{0} y[t+z]  {dz}$).
    \item A parameter of the system (e.g. $\int_{-parameter}^{0} y[t+z]  {dz}$).
    \item An expression beginning with the time variable followed by another expression that can contain both constants, parameters and the time variable. Still, the coordinates in the integral must present a delay written as $Coordinate[timeVariable-expression].$ \end{itemize}
During the development process some choices were made to allow the extended system to further be easily expanded in order to accept other families of delay equations, such as state dependent DDEs and renewal equations (REs).\newline
REs were indeed supported in later development stages, allowing the system inserted to contain zero or more DDEs and ODEs alongside zero or more REs. All the REs must be specified after the DDEs or ODEs, since the evaluation of the right hand side of each equation is subdivided into two macro blocks containing DDEs/ODEs alongside REs. These later development stages regarding the introduction of REs in the software won't be treated in this thesis. 
\newpage
\section{Overview of the extended MatCont GUI}

From the end user perspective the differences with the original version of MatCont GUI can be found when inserting or editing a system. These differences were intended to be both as localized as possible and consistent with MatCont's GUI style. The latter point has been achieved, the first one (i.e. keeping the differences localized in few specific areas) has been satisfied only from the point of view of the end user.\newline
The latter sentence is referred to the fact that while the graphical differences in the interface can be circumscribed in two main windows, the logic and behavior	 of MatCont was modified in several different places (i.e. more than a few files present now different behavior accordingly to the type of system considered). This has been proven necessary to be able to fully support the usage of existing MatCont's components and functionalities with the new extension. This idea is fully expressed by Figure \ref{fig:stackComparison}.
	
\begin{figure}[htp]
\centering
\includegraphics[scale=0.2]{imgTesi/ideaComparison.jpg}
\caption{Comparison between the initial idea and the implementation.}
\label{fig:stackComparison}
\end{figure}

\subsection{The DDE syntax}
\label{syntax}
How the user can insert a DDE equation by typing it with natural syntax (i.e. a syntax similar to how the user would write the system with pen and paper) was a major issue during the extension process, since the main goal was to create a user-friendly interface to study DDE systems and if the syntax would have resulted complex and unnatural, the goal wouldn't have been fully achieved.
\newline
Creating a method that was able to recognise the syntax was a process carried out during the first phase of the development and extended later on. It largely involved the usage of regular grammars\footnote{A formal grammar whose expressive power corresponds to the expressive power of finite automatons}.
\newline
\newline
An equation like $y'(t)=r \cdot y(t)(1-y(t-\tau))$ could be written in the appropriate window (\ref{systemWindow}) by typing \verb| y'=r*y*(1-y[t-TAU])| where TAU is a parameter representing $\tau$ \label{tau}(since only ASCII characters are allowed). 
\newline
In later development stages distributed delay differential equations were added to the equations supported by the software.\newline An equation like $y'(t)=r\cdot \int_{-\tau}^{0} -y(t-DELAY)\,dDELAY $ can be inserted by typing \verb| y'=r * \int_{-TAU}^{0}{-y[t-delay]}{delay}|.
The syntax used to insert integrals is very similar to the syntax used in \LaTeX.
The $\backslash int\_$ string marks the beginning of the integral, then the the first pair of curly braces preceded by the underscore character contains the inferior integration extreme, the second pair of braces preceded by ``\^{ }'' contain the superior integration extreme, the remaining two pair of braces then delimit respectively the function to integrate and the integration variable.
\newline
As it can be seen from the examples above the dependencies of the coordinates on specific points in time must be specified with square brackets and contain as the first term of the expression the time variable, e.g. \texttt{y[t-1]} or \texttt{y[t-TAU]} (where in this latter case \texttt{TAU} is a parameter) followed by an expression describing the delay.

\subsection{Inserting a new ODE or DDE system}
\label{fields}
By proceeding with the insertion of a new system in the extended version of MatCont the user will be presented the window in Figure \ref{fig:insertingWindow}. From this figure we can observe that the interface did maintain its overall structure, indeed it would have been counterintuitive to create a new interface from scratch, mainly for two reasons, the interface resulted to be already well structured and it would have been harder for users that had already used MatCont for years to adapt to a different interface. 

\begin{figure}[htp]
\centering
\includegraphics[scale=0.350]{imgTesi/imgNEW.png}
\caption{MatCont's window to insert a DDE}
\label{fig:insertingWindow}
\end{figure}

From the end user perspective only a few lines of the layout have changed, in the first line line, a dropdown menu lets the user select the type of system to be inserted.
If the option ``ODE'' is chosen then nothing except this first line will be changed, on the other end if ``Delay equation'' is chosen a subwindow asking the user to input two additional parameters (the degree of the collocation polynomial, and the degree used for the polynomial used for the approximation of integrals) will appear.
All the parameters will still be typed in the parameters entry whether they are used to calculate a delay or not. This latter functionality has been modified in later development stages since originally the parameters showing up in the delays (e.g. $\tau$ in \ref{tau}) had to be specified in an additional entry in the subwindow, alongside the two degrees. \newline
The second change the user can observe is that when inserting a system presenting delay equations there is no other option to compute the derivatives other than numerically. Indeed if the user inserted $n$ equations in the appropriate window the system would have $n$ equations from the end user perspective but, internally it will be represented by $n*(M+1)$ equations, where $M$ is the degree of the collocation polynomial. If the user could manually specify the derivatives it would need to write $n*(M+1)$ equations and not just $n$, breaking the abstraction of the discretization process implemented.
\newline
What the user can not see, at least from the graphical interface, is that the binary file associated to a system contains now three additional fields\label{sec:additionalFields}, one describing the type of the system and the other containing, if the system contains delay differential equations, the degree of the collocation polynomial and the degree related to the Clenshaw-Curtis quadrature.
The script file containing the routines used by MatCont when studying the system, if it's related to a ``Delay equation'' system, is also substantially different when comparing it to the same file associated with an ODE system.
The key idea used was based on the same principle adopted when refactoring a method: the methods in the script were semantically modified while keeping the same signature as their ODE counterparts. By doing so it hasn't been proven necessary to modify every part of MatCont invoking those methods.

\begin{figure}[htp]

\hspace*{-6em}
\centering
\includegraphics[scale=1.5]{imgTesi/comparisonedit.png}
\caption{The comparison between MatCont's window to edit an ``ODE'' and ``Delay equation'' system.}
\label{mParameter}
\end{figure}

\subsection{Editing an ODE or DDE system}
The view the user is presented when a system to edit is selected it is still the same of when a new system is inserted, with the exception that now, the dropdown menu that would allow the user to select the system type is locked on the type of the system that is being modified. \newline
When the system considered is a standard ODE system, meaning a system previously inserted with the original version of MatCont, or an ODE system inserted with the extended version of MatCont, the user will be presented the same system details as in the original MatCont that could be modified (i.e. the name, the coordinates, the parameters, the time, and the equations themselves). \newline
On the other hand, if the system considered contained delay equations, the interface will display (as when the insertion of a new ``Delay equation'' system is performed) the subpanel containing the two entries with the degree both the the collocation polynomial and quadrature formula chosen. These values can be modified alongside all the details discussed above for the ``ODE'' systems.


\section{The complete legacy support}

As written in  \ref{fields} the binary file (.mat) associated to a system contains now some additional fields, this could make someone question whether the extended version of MatCont will still work with systems (and thus files) created with the regular version of MatCont (legacy systems). \newline
One of the main goals was indeed to achieve full compatibility with these systems, and this is mainly done when a system is loaded. When MatCont now loads a file checks whether the field denoting the system type in the binary file exists or not. If it finds to be the latter case then the value of the field loaded is set to ``ODE'', since in regular MatCont only these types of systems are allowed (the struct in which the data is loaded contains indeed all the fields, regardless of what structure is saved on the file). From this point in the code everything will work the same as for ODE systems inserted with the extended version of the software.
\newline
\newline
This compatibility with previously created systems can easily be understood from Figure \ref{fig:stateDiagramNew} which describes the states the extended MatCont can go through when inserting the new system. Comparing it with Figure \ref{stateDiagram} the new states are represented by a green rectangle, and it can be observed that the overall structure remains similar while special states for managing the new cases are added. This modus operandi lead also the extension for the other few MatCont's components involved.

\begin{figure}[htp]
\vspace*{0.6em}
\hspace*{-11.5em}
\includegraphics[scale=2.4]{imgTesi/matcont_system_state_diagramNEW2.jpg}
\caption{The state diagram regarding the insertion of a new system in the extended MatCont. The new states are marked by the green border.}
\label{fig:stateDiagramNew}
\end{figure}
\newpage
\section{A deeper view of the parsing process} 

The additional parsing process on the equations of the system, as can be observed from Figure \ref{fig:stateDiagramNew}, takes place only in the case of ``Delay equation'' systems, since otherwise the system won't contain neither discrete nor distributed delays.\newline
This parsing phase is responsible for translating the equations in the system written by the user to the format the system will present in its discretized form in the \emph{fun\_eval} routine in the relative .m file. The \emph{fun\_eval} routine returns a vector containing the evaluation of the system with the specified state and value of the parameters.\newline
The parsing process consist of two main phases, a first routine where each equation of the system is checked for delayed terms (since some equations in a  ``Delay equation'' system may not contain delays) with a regular expression like the one presented below:
\begin{center}
$"coordinateVar\; \backslash [\;timeVar \;\backslash W\;(\;\backslash [\;timeVar \;\backslash W\;[\;\widehat {}\; \backslash ]]*\backslash ]\;|\;[ \;\widehat {} \;  \backslash ]])*\backslash ]"$ 
\end{center}


% \verb$ coordinateVar \int[timeVar \W (\[timeVar \W [^\]]*\]|[^\]])*\W] $ 
%er equivalente in verb

The meaning of the regular expression above applied on an equation of the system can be interpreted as: ``Find in the right hand side of the equation the substrings starting with the specific coordinate, followed by a open square bracket and the time variable, followed by a non alphabetic nor numeric symbol \footnote{A character not in the[a\textendash zA\textendash Z\_0\textendash 9] set.} (expecting the user to insert a $+$ or $-$) followed either by another open square bracket, another non alphabetic nor numeric symbol and a string not containing the close square bracket, or followed by just the latter string. Then repeat this pattern for zero or more times (Kleene star operator) and lastly the substring must end with a close square bracket''.\newline
By doing this it is then possible to extrapolate each delay and replace the substring with another properly built string that contains a function call to a static method defined inside another script file that is responsible for computing the interpolation.
\newline
This process is repeated for each equation with all the different coordinates (since every equation might contain arbitrary delayed terms).
The structure of the regular expression above would already allow the recognition of state dependent equations up to one nested level. \newline \newline
The second routine of the parsing phase, also mainly based on the usage of regular expressions, is responsible for recognising and properly substituting integrals that may appear in the right hand side of an equation. It follows the same principle as the first routine, for each equation, with a properly built regular expression, find all the integrals, recognise and substitute the integration variable and eventually substitute the integral in the right hand side with the correct string representing the approximated integral that will be evaluated in computing \emph{fun\_eval}. These two phases are carried out sequentially, so the string representing the right hand side of each equation in the discretized system produced in output by the first routine is then taken as an input in the second parser routine.
\begin{figure}[htp]
\hspace*{-5.5em}
\includegraphics[scale=0.21]{imgTesi/indeepDiagram.jpg}
\caption{A more detailed state diagram of the ``parsing the equations'' state of Figure \ref{fig:stateDiagramNew}}
\label{fig:stateDiagramNewDetail}
\end{figure}
\newline There are also other secondary less complex routines that write in the .m file other sections of both the \emph{fun\_eval}, \emph{init} and other routines.\newline
One of the routines mentioned is responsible for writing in the file some MATLAB functions each returning a pre-computed matrix. For example one MATLAB function returns the M+1\footnote{where M is the degree of the collocation polynomial} Chebyshev nodes in the interval $[-1,0]$.
\newline \newline
Another one of the secondary routines is responsible for writing in \emph{fun\_eval} an array identified by the name ``delayFunctions'' containing all the delays found in the system. The process used to find the delays is analogous to the one used during the first parsing phase. The array written in \emph{fun\_eval} will be used in every evaluation of the system to determine the maximum (without sign) delay and subsequently scale the Chebyshev nodes (used to interpolate the behavior of the solution in the interval $[-\tau_{max},0]$) from the interval $[-1,0]$ to $[-\tau_{max},0]$.







\newpage
\section{A brief look at the development process}
The extension process followed some of the development principles presented in the Agile Manifesto \cite{agileManifesto}. Focus was especially put on the delivery of working software frequently and on the iterative development process. Overviewing the iteration process the evolution of the software went through these major intermediate phases: \label{iterative}


\begin{itemize}
    \item Development of the GUI extension (without any behavior attached).
    \item Integration of the structure used by MatCont to memorize systems in files and integration of the behavior of the extended interface.
    \item Adjustment of the preprocessing done on the data inserted by the user before passing it to the numerical algorithms (at this point the numerical part of MatCont, such as integration or the continuation of an equilibrium, was working correctly with DDEs with discrete delays).
    \item Integration of MatConts' scripts related to the memorization of the computed data (now the part of Matcont related to plotting the computed curve worked correctly with DDE systems).
    \item Extension of the preprocessing and parsing phase done on the input to recognise distributed delays (i.e. parsing integrals).
    \item Extension of the preprocessing and parsing phase done on the input to recognise renewal equations.
\end{itemize}


Each intermediate version was followed by a small testing phase, consisting primary of Unit testing, a related debugging phase and a code documentation phase. Towards the ending phases of the development some system testing (e.g. higher level testing aiming to check the quality of the whole system \cite{softwareTesting}) took place. These last tests were carried out with ad-hoc dynamical systems and by creating, studying and comparing the results of some of the examples studied with DDE-BifTool\footnote{A software for the bifurcation analysis of delay-differential equations} that can be found on its reference webpage\footnote{http://ddebiftool.sourceforge.net/demos/index.html}.  
\newline
\newline
During the development two main challenges were faced:

\begin{itemize}
	\item Working on a software (MatCont) related to a previously unknown domain to me. This situation can be quite common when dealing with softwares in different fields. How this challenge was faced and overcame was by conducting an introductory study on the main aspects of the field and also by interacting and working with a domain expert.   
	
	\item Understanding and extending sections of code with little or none code documentation (i.e. comments or method contracts) as introduced in section \ref{documentation}. This challenge was mostly faced when adapting the code responsible for handling the curve data. Overall, this challenge has been faced with the help of some of the tools provided by MatLab for the debugging process such as breakpoints and step-by-step execution. 
\end{itemize}

%On the other hand, it has not been a problem adapting to the programming language used by MatLab being a high level programming language and presenting a semantic for the main constructs analogous to many other languages. 
%\newline
\begin{figure}[htp]
\hspace*{-5em}
\includegraphics[scale=0.38]{imgTesi/comparisonMerged.png}
\caption{A qualitative comparison between a Hopf bifurcation continuation curve of DDE-Biftool and Matcont's.}
\label{fig:curveComparison}
\end{figure}

\chapter{Conclusions}
\label{conclusions}
The opportunity of this internship and the work done helped me in gaining a better perspective of how much the job of a computer scientist can vary, while also giving a first glimpse on a scientific field previously unknown to me. Understanding that there is always room for improvement, the software could be further extended in the future, in the next paragraph some of these modifications are presented.

\section{Future extensions}

The system has been extended keeping in mind that other eventual extensions may take place. For example nested integrals or state dependent delay equations (up to some level of nesting) aren't supported yet, but in future these features could be supported and properly managed by a more general extension of MatCont.
The most intuitive way to extend the syntax presented in \ref{syntax} to allow the insertion of state dependent delay equations would be to let the user write them in a form like the following: \verb| y'=y[y[t-TAU]]| where this string represents the equation $y(t)'=y(y(t-\tau))$, so by expressing nested delay dependencies also with square brackets.
\newline
Another possible future integration more related to the GUI and the User experience itself, may be the automatic recognition of the system type once inserted, by doing so, the user wouldn't need to select whether the system inserted is described by ODEs only, or by delay equations. This will bring the GUI even closer to the standard MatCont's GUI.


\newpage
\clearpage
\bibliography{bibliografia}
\bibliographystyle{ieeetr}
\end{document}

